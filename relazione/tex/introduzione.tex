\documentclass[./main.tex]{subfiles}

\begin{document}
\section{Introduzione}
Nella seguente analisi si vuole affrontare un problema di classificazione di attività motorie, in base all'accelerazione a cui è sottoposto un cellulare. I dati sono stati raccolti tramite l'applicazione \texttt{PhonePi}\cite{kumar2019}.\\
Per le $j = 1, \ldots, $ \textbf{(scrivere quante)} attività, si dispongono di $n_j$ misurazioni a intervalli di ampiezza $\Delta t \approx \SI{10}{ms}$ della quantità
$$
\vec{a}_t = \begin{pmatrix}
a_{xt}\\
a_{yt}\\
a_{zt}
\end{pmatrix}\,,
$$
accelerazione del cellulare rispetto agli assi cartesiani di riferimento dell'accelerometro. L'analisi è stata condotta sull'intensità dell'accelerazione $a_t = \|\vec{a}_t\|$, ignorando l'effetto di disturbo dell'accelerazione di gravità.\\
Ciascun segnale è stato suddiviso in intervalli di ampiezza \SI{2}{s}, sui quali sono state calcolate le seguenti variabili riassuntive:
\begin{table}[H]
	\begin{tabular}{ll}
		\texttt{meanA}& Media dell'accelerazione nell'intervallo.\\
		\texttt{maxA}&Massimo dell'accelerazione nell'intervallo.\\
		\texttt{intTrapz}&Integrale del segnale nell'intervallo, calcolato con l'approssimazione per trapezi \textbf{CIT}.\\
		\texttt{MVDeriv}&Media della variazione assoluta della derivata nell'intervallo \textbf{CIT}.
	\end{tabular}
\end{table}
\end{document}
\documentclass[./main.tex]{subfiles}

\begin{document}
\section{Introduzione}
Nella seguente analisi si vuole affrontare un problema di classificazione di attività motorie, utilizzando dati sull'accelerazione a cui è sottoposto un cellulare. Le attività devono essere classificate, con una particolare attenzione nell'identificare correttamente quando il cellulare viene agitato volontariamente ({\em shake}) o meno. I dati sono stati raccolti sullo stesso cellulare, in campioni di circa $\SI{2}{\min}$ per attività, eseguita da più persone diverse. La rilevazione dei dati è stata effettuata tramite l'applicazione \texttt{PhonePi}~\cite{kumar2019}.\\
Per le $j = 1, \ldots, $ \textbf{(scrivere quante)} attività, si dispongono di $n_j$ misurazioni a intervalli di ampiezza $\Delta t = \SI{10}{ms}$ della quantità
$$
\vec{a}_t = \begin{pmatrix}
a_{xt}\\
a_{yt}\\
a_{zt}
\end{pmatrix}\,,
$$
accelerazione del cellulare rispetto agli assi cartesiani di riferimento dell'accelerometro. L'analisi è stata condotta sull'intensità dell'accelerazione $a_t = \|\vec{a}_t\|$, ignorando l'effetto di disturbo dell'accelerazione di gravità. Da ogni campione sono stati tolti i primi e gli ultimi sette secondi, considerando solo i dati per le attività ``a regime''.
In Figura~\ref{fig:esempio} sono riportati dei grafici di esempio dei dati raccolti.
\begin{figure}[H]
	\label{fig:esempio}
	\centering
	\includegraphics[width=.8\textwidth, keepaspectratio]{../../figure/espl.png}
	\caption{Misurazioni nell'intervallo $ [\SI{10}{\second}, \SI{30}{\second}] $ per le attività motorie di salita e discesa di scale, salti, shake, corsa con cellulare in tasca.}
\end{figure}
\end{document}
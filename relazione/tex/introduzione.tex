\documentclass[./main.tex]{subfiles}

\begin{document}
\section{Introduzione}
Nella seguente analisi si vuole affrontare un problema di classificazione di attività motorie, utilizzando dati sull'accelerazione a cui è sottoposto un cellulare. Le attività devono essere classificate con particolare attenzione nell'identificare correttamente quando il cellulare viene agitato volontariamente ({\em shake}) o meno. I dati sono stati raccolti sullo stesso cellulare da persone diverse, tramite l'applicazione \texttt{PhonePi}\cite{kumar2019}.\\
Per le $j = 1, \ldots, $ \textbf{(scrivere quante)} attività, si dispongono di $n_j$ misurazioni a intervalli di ampiezza $\Delta t \approx \SI{10}{ms}$ della quantità
$$
\vec{a}_t = \begin{pmatrix}
a_{xt}\\
a_{yt}\\
a_{zt}
\end{pmatrix}\,,
$$
accelerazione del cellulare rispetto agli assi cartesiani di riferimento dell'accelerometro. L'analisi è stata condotta sull'intensità dell'accelerazione $a_t = \|\vec{a}_t\|$, ignorando l'effetto di disturbo dell'accelerazione di gravità.\\
\end{document}
\documentclass[./main.tex]{subfiles}

\begin{document}
\section{Albero di classificazione}
Dopo aver suddiviso le variabili in insieme di stima e insieme di validazione, si è utilizzato il primo insieme per la crescita dell'albero. I dati sono stati suddivisi in modo tale da avere il 25$\%$ delle osservazioni nell'inisieme di validazione.
L'albero è stato fatto crescere completamente utilizzando l'entropia come misura di impurità. 

Dato che l'obiettivo è quello di riuscire a discriminare al meglio tra shake e altre tipologie di movimento si è utilizzata una funzione di perdita. Tale funzione pesa maggiomente ..... per diminuire l'errore di confondere altri movimenti per shake.


%L'algoritmo di potatura è stato eseguito sull'insieme di validazione utilizzando la devianza come funzione obiettivo da minimizzare. Dal grafico ?? si nota che la devianza minima è ottenuta con .... foglie.



\end{document}
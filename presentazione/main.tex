\documentclass{beamer}
\usetheme{Padova}
\usepackage{presentazione}

\title{\textsc{Classificazione di attività motorie tramite accelerometro del cellulare}}
\author{Galtarossa Luisa\and Grassi Alberto\and Montin Anna\and Zago Daniele}
\date{}

\begin{document}
\begin{frame}
\titlepage
\end{frame}

\section{Introduzione}
\begin{frame}{Il problema analizzato}
Ormai con il cellulare è possibile fare moltissime cose. 
Cancellare messaggi, avere più giga...è tutto possibile con un solo shake.\\
\smallskip
Ma come possiamo riconoscere uno shake?
\begin{figure}[H]
\includegraphics[width=0.4\textwidth]{./images/vodafoneshake.png}
\end{figure}
\end{frame}

\begin{frame}{La nostra idea}
Riconoscere uno shake tra diverse attività motorie.\\
\smallskip
\begin{columns}[T] % align columns
\begin{column}{.48\textwidth}
Le attività analizzate sono:
\begin{itemize}
\item Camminata;
\item Camminata tasca;
\item Corsa;
\item Corsa tasca;
\item Utilizzo a riposo;
\item Salti;
\item Salita e discesa di scale.
\end{itemize}
\end{column}%
\hfill%
\begin{column}{.4\textwidth}
\begin{figure}[H]
\includegraphics[width=0.6\textwidth]{./images/attivit.jpg}
\end{figure}
\end{column}%
\end{columns}
\end{frame}

\begin{frame}{Raccolta dati}
I dati sono stati raccolti tramite l'applicazione \cite{kumarPhonePiSampleServer2019}.\\
\smallskip
Questa fornisce l'accelerazione in forma vettoriale ad ogni istante $t$.
\[
\vec{a_t} = \begin{pmatrix}
a_x \\ a_y \\ a_z
\end{pmatrix}
\]
\end{frame}

\begin{frame}{Accelerazione}
L'accelerazione viene calcolata come $a_t =\sqrt{a_x^2+a_y^2+a_z^2}$.
\begin{table}[H]
\begin{tabular}{cccccccc}
\toprule
y & $a_0$ & $a_1$ & $a_2$  & $\dots$ & $a_{149}$\\
\midrule
camminata & 5.449 & 5.300 & 5.344 &  $\dots$ & 13.147\\

camminata & 13.977 & 14.910 & 15.567 &  $\dots$ & 5.480\\

camminata & 5.608 & 5.868 & 6.143 &  $\dots$ & 18.227\\

camminata & 19.026 & 18.886 & 18.098 &  $\dots$ & 6.299\\

$\vdots$ & $\vdots$ & $\vdots$ & $\vdots$ &  $\ddots$ & $\vdots$\\

shake & 11.480 & 14.663 & 16.968 &  $\dots$ & 8.474\\

\bottomrule
\end{tabular}
\end{table}
Ma cosa rappresenta questa tabella?
\end{frame}

\section{Esplorative}
\begin{frame}
grafici
\end{frame}


\begin{frame}

\begin{columns}[T] % align columns
\begin{column}{.48\textwidth}
\begin{figure}[H]
\includegraphics[width=\textwidth]{../figure/ICA.png}
\end{figure}
\end{column}%
\hfill%
\begin{column}{.48\textwidth}
$f(x)=x^2$
\end{column}%
\end{columns}
\end{frame}
%\documentclass[./main.tex]{subfiles}

\begin{document}
\section{Preprocessamento ed analisi esplorativa}
Ciascun segnale è stato suddiviso in intervalli di ampiezza \SI{1.5}{s}, che sono stati utilizzati come unità statistiche. Dal momento che ciascuna unità è una serie storica, si è scelto di riassumerla con le seguenti variabili:
\begin{table}[H]
	\centering
	\begin{tabular}{ll}
		\texttt{minA}& minimo dell'accelerazione\\
		\texttt{medA}& mediana dell'accelerazione\\
		\texttt{varA}& varianza dell'accelerazione\\
		\texttt{maxA}& massimo dell'accelerazione\\
		\texttt{meanA}& media dell'accelerazione\\
		\texttt{MVDeriv}& media del valore assoluto della derivata seconda.
	\end{tabular}
\end{table}
Siccome i segnali di {\em shake} sono ad alta frequenza, si è scelto di utilizzare la variabile \texttt{MVDeriv} per discriminare le alte frequenze dalle basse frequenze. Infatti, \texttt{MVDeriv} assume valori grandi quando la curvatura della funzione è mediamente elevata nell'intervallo, cioé quando la frequenza è alta. L'operazione di derivazione all'istante $t$ è approssimata con\cite{NumpyGradientNumPy}
\[
\dot{a}_t = \dfrac{a_{t + 1} - a_{t - 1}}{2}\,.
\]
I dati sono stati suddivisi in insieme di stima (70\%), di convalida (15\%) e di test (15\%). Le previsioni riportate nelle conclusioni sono relative all'insieme di test, su cui è stato applicato il modello migliore valutato in termini di accuratezza sull\rq{}insieme di convalida.
In appendice sono riportate le distribuzioni marginali delle variabili riassuntive per ciascuna classe.\\

Per valutare graficamente la separabilità delle classi, l'informazione migliore è data dal t-SNE sui dati sbiancati (Figura~\ref{fig:tsne}). Le componenti principali e le componenti indipendenti, invece, non riescono a separare le classi in modo soddisfacente (Appendice A).
\begin{figure}[H]
	\centering
	\includegraphics[width=.8\textwidth]{../../figure/t-SNE.png}
	\caption{{ t-SNE per i dati sbiancati}}
	\label{fig:tsne}
\end{figure}
\end{document}
%\input{tex/...}

%\printbibliography

\end{document}